% Nome do capítulo
% Diminuir espaçamento entre título e texto
\vspace{-1.9cm}

\chapter{Conclusão e trabalhos futuros}
\label{conslusao-e-trabalhos-furutos}

  Neste capítulo são apresentadas as considerações finais, as contribuições do
  trabalho e as perspectivas futuras.


\section{Considerações finais}
% Label para referenciar
\label{conslusao-do-trabalho}


% Texto do capítulo
  Conforme a pesquisa do \cite{IBGE:2013} número de usuários conectados a Internet tem crescido bastante e é necessário
  que os aplicativos para Web sejam capazes de atender à essa demanda. Neste trabalhos propomos a utilização da t
  ecnologia Node.Js e suas características para solucionar o problema de milhares de conexões em servidores Web.
  
  Nos testes realizados com a ferramenta Loader.io foi possível mostrar a capacidade da tecnologia em responder as requisições
  feitas ao protótipo em um tempo bem inferior ao seu protótipo comparativo. E a média de número de sucessos nos 03 tipos de 
  teste foi superior ao protótipo feito em Django. Além do fator desempenho citado anteriormente a plataforma Node.Js foi capaz
  de receber maiores cargas de usuários em um servidor Web com poucos recursos de \textit{hardware} sendo que no primeiro teste o
  Node.Js conseguiu responder a 1000 clientes por teste contra 4000 clientes por teste do protótipo Django; No segundo teste
  400 usuários por segundo contra apenas 100 usuários no segundo protótipo e no último teste de carga a plataforma respondeu
  ao minímo de 450 a 500 usuário contra apenas 300 a 350 do protótipo comparativo. Considerando a natureza deste último, como
  sendo um teste de carga, o Node.Js em um servidor Web com poucos recursos de \textit{hardware} escalou e respondeu melhor
  os usuários conectados ao sistema.
  
  Voltando ao fator desempenho, no primeiro teste o Node.Js teve sua média de tempo de resposta entre 12 a 14 milissegundos, sendo 
  sua máxima 13.71 milissegundos respondendo a 10000 requisições com sucesso, contra 6771 milissegundos respondendo a apenas 4000
  requisições com sucesso. No segundo teste o Node.Js teve a média do tempo de resposta inferior a  1.9 segundos com 400 usuários
  e a API em Django respondeu a apenas 100 usuários com a média de 4.5 segundos. Com estes dados é possível validar o desempenho
  do Node.Js, objeto de pesquisa, contra os aplicativos Web ditos tradicionais.
  
  Um dos fatores a serem considerados, foi no desenvolvimento da aplicação, pois com o framework Django tem-se um rápido 
  desenvolvimento do software capaz de fornecer uma API robusta com poucas linhas de código. No framework Express.Js o desenvolvimento
  é custoso para desenvolvedores menos experientes pois se trata de uma nova abordagem e paradigma. 
  O outro ponto a ser considerado é a maturidade do framework Django, desde de 2005, e da linguagem Python. O Node.js
  nasceu em 2009 e apesar de ter uma ampla comunidade \textit{open source} que contribui com código e bibliotecas complementares
  ao sistema ha muitas linhas de código a serem escritas. 
  
  Uma melhoria em nível do protótipo Django seria a utilização de cache de consultas e dos resultados em JSON que poderiam
  reduzir o tempo de resposta e aumentar o número de sucessos. Este cache pode ser colocado no \textit{framework} e utilizar também
  softwares memcached e/ou varnish em um outro servidor.\cite{UsandoDjango:2013}
  Também é valido considerar a troca do \textit{framework} Django para o Tornado\footnote{Web framework e biblioteca de rede 
  assíncrona. Disponível em http://www.tornadoweb.org/en/stable/} escrito em Python conforme a apresentação de Rafel 
  Martins com o título \textbf{Bastidores da API de esportes da Globo.com: 4 milhões de reqs/dia com Python/Tornado, Redis e Nginx}.
  Esta última tecnologia apresentada é um ótimo objeto de estudo e que poderia comparar as duas tecnologias, Node.Js e Python, pois 
  o Tornado possui características não bloqueantes e assíncronas.\cite{TornadoGloboEsporte:2014}

  Após a implementação de todas essas funções ou até mesmo a troca de tecnologia do protótipo comparativo, 
  deverá ser executado novamente o plano de teste e o mesmo ambiente de teste para que não altere os resultados. Com estes
  resultados é posśivel identificar os impactos causados pelas modificações efetuadas no desenvolvimento do software.
  
\section{Trabalhos futuros}
% Label para referenciar
\label{trabalhos-furutos}
  
  Este trabalho pode ser considerado um ponto de partida inicial para conhecer a tecnologia e ambiente desenvolvimento de aplicativos 
  para Web, com foco no desempenho. Assim como as possibilidades de continuação e refinamentos deste trabalho 
  são variadas o Node.Js pode ser explorado em diversos outros contextos, tais como aplicações em tempo real, stream de dados dentre 
  vários outros usos.
  
  No Brasil o Node.Js é pouco dinfundido e requer de mais estudos e materiais em língua portuguesa para os desenvolvedores iniciantes
  e empresas que desejam superar barreiras de escalabilidade com software. 
  
  De fato iremos continuar a acompanhar a evolução da plataforma Node.Js e assim como muitas empresas no exterior ja o fizeram e 
  podem ser vistos na documentação do Node.Js \citeonline{JoyentProjects:2014}.
  
  

