% Nome do capítulo
\chapter{O ambiente Node.Js}
% Label para referenciar
\label{ambiente-node-js}

% Diminuir espaçamento entre título e texto
\vspace{-1.9cm}

% Texto do capítulo


%  Para resolver este problema adota-se o paradigma de programação orientada a eventos utilizando
%  as caracteriticas nativas do JavaScript: modelo de eventos assíncronos, funções anônimas e \textit{callbacks}. \cite{Junior:2012}
  
%  Como \citeonline{Junior:2012} exemplificou, um programa assíncrono ao fazer uma requisição 
%  a um banco de dados especifica o que deve ser feito com os resultados do banco de dados, ele não espera a 
%  finalização da requisição e continua a processar outras atividades existes no programa. 
%  Apenas quando o resultado da requisição é retornado do banco de dados, a codificação para manipular os estes dados 
%  é executado. A esta lógica de programação, executada após todos os dados serem retornados, dá-se ao nome de callback.\cite[p. 2]{Junior:2012} 
  
  
%  O Node.Js afirma que ele nunca irá ter bloqueios ou impasses, já que bloqueios não é uma característica 
%  da sua plataforma mesmo em processamento de entradas e saídas e que um servidor pode suportar 
%  dezenas de milhares de conexões simultâneas.\cite{Abernethy:2011}

  Criado por Ryan Dhal em 2009, a plataforma Node.Js tem como objetivo mitigar o problema descrito no capítulo \ref{introducao}
  e prover um ambiente de programação amplo e forte para os desenvolvedores. 
  
  Nas próximas seções será explicado ao leitor as principais características do ecosistema do Node.Js.
 
  
%  Conforme descrito no capítulo introdutório \cite{Junior:2012} descreve o Node.Js como uma plataforma orientada a 
%  eventos não bloqueantes projetada para construção de aplicações de rede rápidas e escaláveis utilizando o JavaScript.
  
%  Neste contexto do ambiente Node.Js o JavaScript é uma das linguagens de programação mais utilizadas em interfaces de sites, e o Node.Js
%  permite utilizar essa linguagem de programação e aplicá-la em outros contextos como servidores web. Possui um alto
%  desempenho através do JavaScript Enginie V8, do Google.\cite{Hughes:2012}.
  
%  O V8 utiliza umas das técnicas recentes de compiladores que permite que o código escrito em linguagem de alto nível,
%  tal como o JavaScript, execute de forma semelhante a linguagens de baixo nível como C. O Node.Js também aproveita
%  do paradigma de orientação a eventos da linguagem e retira proveito disso para produzir servidores escaláveis usando 
%  uma arquitetura chamada de ciclo de eventos, do inglês, Event Loop.\cite{Hughes:2012}.
  
%  Node.Js é extensível com grandes módulos construídos por uma comunidade ativa desde 2009 quando foi criado
%  por Ryan Dhal.
  
\section{Programação Orientada a Eventos}
\label{programacao-orientada-a-eventos}

  A programação orientada a eventos é a principal abordagem para o sucesso do Node.Js. Sendo escolhida para
  minimizar os impactos de alta concorrência descrito anteriormente no capítulo \ref{introducao}. 
  
  As aplicações criadas com o Node.Js tem como núcleo e referência os eventos. Estes eventos são indicações de que ocorreu algo, 
  havendo dois atores para este modelo de programação.\cite{Oliveira:2012}
  \citeonline{Junior:2012} utiliza a terminologia de produtor do evento (\textit{event producer}) 
  e o consumidor do evento (\textit{event consumer}) para identificar estes dois atores. Em contrapartida os servidores
  web tradicionais utiliza o conceito de ação e resposta.
  
  
  \begin{figure}[H]
    % Alterar espaçamentos antes e depois do caption
    \setlength{\abovecaptionskip}{0pt}
    \setlength{\belowcaptionskip}{0pt}
    % Caption
    \caption[Ciclo de eventos no Node.Js]{Ciclo de eventos no Node.Js}
    \centering
    \includegraphics[width=.85\textwidth]{imagem/node-js-system-twitter-BusyRich.png}
    % Caption centralizada
    \captionsetup{justification=centering}
    \captionfont{\small{\textbf{\\Fonte: \cite{NodeSystem:2014}}}}	
    \label{fig:node-js-system-loop}
  \end{figure}
  
  A figura \ref{fig:node-js-system-loop} exibe uma síntese do sistema Node.JS.
  
  Para atender a necesidade dessa abordagem o JavaScript casou-se muito bem com a orientação a eventos, pois provê nativamente
  o modelo de eventos assíncronos em operações de entrada e saída, além de suporte de callbacks.\cite{Oliveira:2012}

  \citeonline{Pereira:2013} compara que a orientação a eventos do Node.Js se espelha na filosofia de orientação 
  a eventos utilizado nos navegadores; a diferença entre eles é que no Node.Js 
  não existe eventos de clique do mouse, teclas pressionadas do teclado (keyup) ou qualquer evento de componentes HTML. Mas operações
  de entrada e saída do servidor, assim como eventos de conexão ao banco de dados, abertura de arquivo e \textit{streaming}
  de dados.
  
  A principal diferença em sistemas baseados em eventos, é que o produtor do evento não espera pela ação a ser executada
  pelo servidor. \cite{Junior:2012}    

  Em complemento a esta abordagem o Node.Js inicia seu objetivo de criar aplicações de rede escaláveis, utilizando thread única
  e o cilco de eventos para resolver os galargalos de altas conexões.

\section{Única thread e o cliclo de eventos}
\label{single-thread}

  Uma thread pode ser conceituada como um conjunto de instruções executadas pelos processos do sistema operacional
  
  Em Node.Js as conexões são recebidas em uma única thread invocada pelo nó de servidor de processos \textit{node server process}.
  \citeonline{Abernethy:2011} cita que ao invés de criar novas \textit{threads}
  no sistema operacional para cada conexão e alocar a memória RAM que acompanha essas \textit{threads}, 
  cada conexão dispara um evento executado no processo do motor Node.Js.
  
  Isso ocorre porque ao ser invocado um nó de servidor de processos (node server
  process), ele roda apenas em 01 (uma) thread que suporta um alto número de conexões, isso é
  possível, pois há um loop implícito que cobre o código, esse loop é denominado de event loop
  e tem como função esperar os eventos e repassá-los ao manipulador de eventos. \cite{Tilkov:2010}
  
  \cite{Powers:2012} cita também o single thread como um dos benefícios do ambiente do Node.Js 
  pois o aplicativo pode ser facilmente escalável uma vez que em um único segmento de execução não ha uma enorme 
  sobrecarga de requisições. Citando o exemplo de seu livro, ao criar uma aplicação em \ac{PHP} semelhante 
  à aplicação Node.Js o usuário veria a mesma página, mas ao visualzia os processos desta aplicação haverá uma 
  diferença.
  
  Este aplicativo \ac{PHP} no servidor web Apache, cada pedido que for solicitado irá abrir um 
  processo filho do Apache. Em servidores menos otimizados a capacidade de criar processos filhos
  restringe-se a par de centenas de processos filhos em paralelo. Se a por ventura a quantidade de solicitações
  for maior que capacidade de processos filhos do Apache, o cliente entrará numa fila e esperar por uma resposta.\cite{Powers:2012}

  Assim como \citeonline{Powers:2012}, \citeonline{Hughes:2012} afirma que o conceito de thread única é importante para 
  toda a plataforma do Node.Js, porém é uma das críticas feitas ao Node.Js pela comunidade de desenvolvedores 
  que utilizando este conceito não é possivel realizar concorrência no nó de servidor de processos.
  
  \citeonline{Pereira:2013}, enfatiza que não é possível trabalhar com programação 
  concorrente em plataforma multi thread nativamente com o ambiente Node.Js. Mas existem maneiras de implementar sistemas concorrentes, 
  como por exemplo, a utilização de clusters, o qual é um módulo nativo do ambiente Node.Js.
  
  Além disso, é possível compartilhar sockets através de bibliotecas de terceiros como o multi-node \citeonline{Zyp:2010}.
  Desta maneira o nó de servidor de processos pode ser instanciado em um ou mais processadores, oferencendo paralelismo, 
  atendendo e recebendo requisições nas mesma porta. Cabendo ao sistema operacional atuar como balanceador de carga.\cite{Oliveira:2012}
  
  Em conjunto com a thread única o ambiente Node.Js possui o cilo de eventos sendo o agente responsável por escutar e 
  emitir eventos dentro do sistema. Onde o ciclo de eventos é uma repetição infinita que a cada interação verifica em sua 
  fila de eventos se um determinado evento foi emitido ou se existem novos eventos. Estes eventos só aparecem na 
  fila quando são emitidos durante as suas interações na aplicação; quando ocorre, é emitido um evento, então este evento 
  é executado e enviados para a fila de executados.\cite{Pereira:2013}
  
  \cite{Pereira:2013} diz que a programação orientada a eventos do Node.Js 
  foi inspirado pelos \textit{frameworks} Event Machine\footnote{http://rubyeventmachine.com/} do 
  Ruby\footnote{https://www.ruby-lang.org/en/} e Twisted\footnote{https://twistedmatrix.com/trac/} do 
  Python\footnote{https://www.python.org/}, porém o ciclo de eventos do Node.Js é mais perfomático pois seu mecanismo 
  é nativamente executado de forma não bloqueante sendo o diferencial em relação a outros ambientes de programação.
  
  \cite{Wilson:2013} enaltece os eventos do Node.Js, fornecido pelo JavaScript. Em outras linguagens de programação 
  lidam com fluxos de trabalho em threads múltiplas e concorrentes, onde cada thread gasta a maioria de seu tempo aguardando 
  operações bloqueadoras de entrada e saída, tais como: leitura ou escrita em disco, manipulação do banco de dados 
  acesso a informações pela rede. O que não ocorre com o Node.Js.
   
  Intuitivamente o ciclo de eventos é associado a vida cotidiana onde cada solicitação de evento necessita de um retorno. 
  O desenvolvimento em Node.js visa que escreva o código e seus metódos trabalhe com uma chamda de retorno de cada vez, 
  o programa será compreensível e também capaz de executar rapidamente várias tarefas de forma eficiente.\cite{Hughes:2012}
  
  Possuindo uma ideia de como o ciclo de eventos funciona, o desenvolvedor é capaz de usá-lo em toda sua potencialidade, 
  conseguindo vantagens e evitando armadilhas dessa abordagem.\cite{Hughes:2012}
  
  Aprofundando \cite{Pereira:2013} cita o EventEmitter como módulo responsável por emitir estes eventos e em 
  grande maioria das bibliotecas do ambiente Node.Js utiliza as funcionalidades de eventos deste módulo. 
  No processo de execução do evento pode-se programar qualquer lógica de programação através do 
  mecanismo de chamada de retorno. Tal chamada de retorno pode ser executado através de uma função de escuta, 
  semanticamente conhecida pelo on().
  
  Essa seção é bem descrita e exemplificada por \cite{Wilson:2013} em seu livro que nos mostra o uso e o 
  desenvolvimento de eventos.

  Deste modo, um servidor Node.js pode suportar dezenas de milhares de conexões simultâneas, 
  pois ele altera todo o contexto do servidor e o único gargalo passa a ser a capacidade de tráfego 
  de um sistema e não mais o número de conexões.\cite{Abernethy:2011}
  
\section{Assíncronismo, Chamadas de retorno e \textit{callback's hell}}
\label{chamadas-de-retorno-e-callback-hell}

  De acordo com \citeonline{Wilson:2013} o JavaScript utiliza de \textit{callbacks}
  para abordar o problema a  partir do lado oposto; ao invés de gerenciar processos de execução prolongada, 
  os desenvolvedores associam eventos específicos e escrevem funções especiais, chamadas \textit{callbacks}
  (chamadas de retorno), que são executadas quando o critério do evento é atingido.

\subsection{Evitando \textit{callbacks hell}}

  \cite{Pereira:2013} relata que o JavaScript possui boa performance trabalhando de forma assíncrona porém em certos 
  momentos do desenvolvimentos, inevitavelmente será implementado diversas funções assíncronas encadeadas umas nas 
  outras através das suas funções \textit{callbacks} criando o \textit{callbacks hell}.
  
  No código \ref{leitura-arquivos-diretorio-node} implementado por \citeonline{Pereira:2013}, e referenciado no Anexo \ref{primeiro-anexo}
  ha uma simples leitura de arquivos de um diretório qualquer sendo impresso o nome do arquivo e seu tamanho em
  bytes. Este exemplo feito pelo autor demonstra que uma simples tarefa possui várias chamadas de retorno encadeadas. Também
  é questionado como seria a organização caso a solução do problema fosse mais complexa. Pode-se entender que tal código
  seria um caos e de difícil manutenabilidade.
  
  Também é afirmado, que, a linguagem JavaScript ser assíncrona resulta em ganhos de 
  performace, ha o problema de perda de controle do que está sendo executado, acesso à variáveis devido à troca de escopos
  neste emaranhado de chamadas de retorno.
  
  Um ponto a se atentar, é que, que as chamadas de retorno no Node.Js possuem como parâmetro uma variável de erro. Se existir
  este parâmetro, é recomendado por \citeonline{Pereira:2013} que realize primeiramente o tratamento deste erro na execução da função
  impossibilitando a execução aleatória quando surgir tal erro.
  
  Uma das maneiras de se evitar o temido \textit{callback hell}, e dado que é, uma boa prática de codificação JavaScript é
  criar funções que expressem seu objetivo de forma isoladas, salvando os retornos em váriaveis e passando-as em outros
  chamadas de retorno como parâmetros. A organização do código pode ser visto no código \ref{leitura-arquivos-diretorio-node-callback-heaven} do
  Anexo \ref{primeiro-anexo}.\cite{Pereira:2013}
 
  Uma abordagem utilizada pela empresa StrongLoop é a utilização do módulo async \footnote{https://github.com/caolan/async},
  sendo o mais popular entre os desenvolvedores e também fica mais próximo do \textit{core} (núcleo) do Node.Js. Este módulo
  possui o metódo async.waterfall que provê um controle em série, em que os dados podem ser passados para a próxima função
  usando o parametro next. O metódo async.map executa o comando fs.stat (buscar status do arquivo) do Node.Js sobre uma matriz
  de caminhos, em paralelo. Em seguida retorna uma matriz com a ordem mantida dos resultados. Como dito pela empresa
  Strongloop, este módulo garante que somente uma chamada de retorno será retornada, propagação de erros e controle do 
  paralelismo automáticamente. 
  
  Outro módulo é apresentado pela empresa Strongloop é utilização de \textit{Promises} que fornece tratamento de erros
  e regalias de programação funcional. Para tal, é necessário utilizar o módulo Q \footnote{https://github.com/kriskowal/q}
  que através do metódo q.all executa todas as chamadas de status dos arquivos em paralelo e em seguida retorna uma matriz
  com a ordem dos dados mantidas. Ao contrário de exemplos anteriores, quaisquer exceção é lançada dentro da cadeia dos
  \textit{promises}. Somente depois tais exceções são capturadas e manipuladas.
  
  Por fim como descreve a \citeonline{Strongloop:2013} existe a abordagem utilizando \textit{generators} que estarão
  contemplatos e integrados oficialmente em versões posteriores à 0.11.2 do Node.Js. Os \textit{generators} podem ser definidos
  como co-rotinas leves para o JavaScript. Estes \textit{generators} permitem que uma função possa ser suspensa e retornada
  utilizando a palavra reservada yield. A empresa recomenda utilizar o módulo CO \footnote{https://github.com/visionmedia/co},
  mas nada impede a utilização de outros módulos. Ao utilizar o módulo CO é possível manipular erros (incluindo exceções levantadas)
  serão passadas para a função de chamada de retorno. Também é habilitado o uso de blocos \textit{try/catch} em torno das 
  declarações yield.
  
  A empresa \cite{Strongloop:2013} investigou três possibilidades de mitigar o problema dos \textit{callbacks hell}, com o 
  intuito de obtenção de controle do fluxo da aplicação. Um interesse maior surgiu pela a abordagem dos \textit{generators}
  apesar de não empregar em seus projetos. Independente de qual módulo e abordagem for utilizada ela reafirma que é recomendado
  utilizar a modularização em qualquer parte da aplicação e bibliotecas descritas (\textit{async,promises, generators}).
  
  Todo os código e comentários podem ser vistos no artigo da empresa. \cite{Strongloop:2013}
  
  
\section{Por que usar assíncrono}
\label{porque-usar-assincrono}

  No ambiente de desenvolvimento Node.Js é importante entender e saber trabalhar com as chamadas assíncronas. 
  \cite{Pereira:2013} em seu livro exemplifica em código as diferenças entre uma função síncrona e assíncrona 
  em relação ao tempo em que são executadas. Este código nada mais é que uma repetição de 5 interações e a cada 
  iteração desta repetição será criado um arquivo texto.
  
  Veja o tempo gasto utilizando o modelo síncrono da função \textit{fs.writeFileSync}.
  
  \begin{figure}[H]
  % Alterar espaçamentos antes e depois do caption
  \setlength{\abovecaptionskip}{0pt}
  \setlength{\belowcaptionskip}{0pt}
  % Caption
  \caption[Tempo de resposta, metódo síncrono bloqueante]{Tempo de resposta, metódo síncrono bloqueante}
  \centering
  \includegraphics[width=.85\textwidth]{imagem/timeline-node-sync-caio-ribeiro.png}
  % Caption centralizada
  \captionsetup{justification=centering}
  \captionfont{\small{\textbf{\\Fonte: \cite{Pereira:2013}}}}	
  \label{fig:timeline-sync}
  \end{figure}
  
  A Figura \ref{fig:timeline-sync} exibe um quadro com um tempo de 200 ms de resposta para cada um dos 5 arquivos.

  \begin{figure}[H]
  % Alterar espaçamentos antes e depois do caption
  \setlength{\abovecaptionskip}{0pt}
  \setlength{\belowcaptionskip}{0pt}
  % Caption
  \caption[Tempo de resposta, metódo assíncrono não-bloqueante]{Tempo de resposta, metódo assíncrono não-bloqueante}
  \centering
  \includegraphics[width=.85\textwidth]{imagem/timeline-node-async-caio-ribeiro.png}
  % Caption centralizada
  \captionsetup{justification=centering}
  \captionfont{\small{\textbf{\\Fonte: \cite{Pereira:2013}}}}	
  \label{fig:timeline-async}
  \end{figure}

  
  Agora na Figura \ref{fig:timeline-async} o tempo total de duração foram de 200 milesegundos, pois foram executados
  de forma assíncrona maximizando o processamento.

\section{Framework Express}
\label{framework-express}

  \cite{Powers:2012} descreve que no geral um \textit{framework} ajuda na infraestrutura e que nos permite criar sites e aplicações
  com agilidade, fornecendo ao desenvolvedor um esqueleto capaz de oferecer um suporte no processo de desenvolvimento de
  software. Com os \textit{frameworks} foca-se na criação de funcionalidades da nossa aplicação ou site e 
  também fornece coesão ao código, o que nos beneficia com legibilidade e manutenabilidade.

  \cite{Pereira:2013} complementa que utilizar a API HTTP nativa do Node.Js pode ser um processo moroso e desgastante
  para o desenvolvedor. 
  Conforme surgem novas necessidades de implementação e novas funcionalidades serão acrescidas,
  os códigos se tornarão gigantescos aumentando a complexidade do projeto e dificultando futuras manutenções.
  
  Assim surge o \textit{framework} Express.Js para solucionar necessidades e agilizar no desenvolvimento.
  
  \citeonline{Powers:2012} compara que o \textit{Express.Js} é parecido com o \textit{framework} Sinatra porém bem mais \textit{RESTFUL}. 
  Pereira(2012) reafirma e complementa que o módulo \textit{Express.Js} foi inspirado pelo \textit{framework} Sinatra da 
  linguagem Ruby e que é bastante utilizado em aplicações web de grande escala.
  
  Suas características são descritas por \citeonline{Pereira:2013}:
  
  \begin{compactitem}
    \item[a)] \ac{MVR};   
    
    \item[b)] \ac{MVC};
    
    \item[c)] Roteamento de \ac{URL} com chamadas de retorno;
    
    \item[d)] Middleware;
    
    \item[e)] Interface \ac{REST};
    
    \item[f)] Suporte a File Uploads.
    
    \item[g)] Configuração baseada em variáveis de ambiente;
    
    \item[h)] Suporte a helpers dinâmicos;
    
    \item[i)] Integração com Templates Enginies;
    
    \item[j)] Integração com SQL e NoSQL;
    
  \end{compactitem}
  
  Ao criar um esqueleto utilizando o \textit{Express.js} é importante ter conhecimento do que cada
  arquivo ou diretório representa. \cite{Powers:2012, Hughes:2012} não apresentam um descritivo de cada arquivo 
  ou diretório e seu papel. Entretanto \cite{Pereira:2013, Wilson:2013} aprofundam mais neste assunto que podem ser vistos
  no \ref{apend:express-skel}.
  
  O arquivo de \textit{package.json}, de acordo com \citeonline{Wilson:2013} sempre é necessário ser criado 
  em seu projeto e que ele é responsável por fornecer detalhes sobre as condições de operação e configuração 
  esperadas por seu código. \cite{Wilson:2013} complementa que este arquivo ajuda a prevenir que alterações 
  futuras em módulos de terceiros quebrem a lógica da aplicação.

  No livro, \cite{Wilson:2013} exibe um exemplo do arquivo 
  \textit{package.json} o qual é utilizado para sincronizar a aplicação com dependências, sendo importante associar 
  a aplicação a uma versão especifica. 
  
  Para maiores detalhes sobre as notações semânticas utilizada pelo \ac{NPM} visite a documentação \cite{Semver:2013}.

\subsection{O servidor web com \textit{Express.Js}}
\label{servidor-web-express-js}

  
  De acordo com \cite{Wilson:2013} o interessante do Node.Js é que o código do 
  programa que se escreve para ele também é a implementação do servidor. 
  Seguindo este modelo tem-se a expectativa de que a aplicação funcione e se comporte de modo semelhante 
  ao ambiente de produção assim como no desenvolvimento, pois não existe nenhuma biblioteca, nenhum intermediário 
  ou \textit{daemon} que esteja no caminho.
  
  O arquivo app.js criado pelo \textit{Express.js} é em suma pequeno mas com grandes funcionalidades inclusas como 
  roteamento para solicitações \ac{HTTP} entrantes, motores de visões para renderizar marcações do HTML5
  e também download dos arquivos estáticos.