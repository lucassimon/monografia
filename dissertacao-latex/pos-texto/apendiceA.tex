%------------------------------------------------------------------------------------------------------------------------------------------------------
% Reiniciar numeração das figuras que aparecem no apêndice
\setcounter{figure}{0}

\chapter{Primeiro apêndice}
\label{apend:express-skel}

% Para diminuir espaçamento entre o título e o texto
\vspace{-1.9cm}

\begin{compactitem}
  \item[a)] \textit{package.json}:
  
  \cite{Pereira:2013} diz que este arquivo contém as principais informações sobre a aplicação como: 
  nome, autor, versão, colaboradores, URL, dependências e outros.
  
  \item[b)] \textit{public}:
  
  Pasta pública que armazena código estático como imagens, css e javascript. \cite{Pereira:2013}
  
  \item[c)] \textit{app.js}:
  
  \cite{Wilson:2013} descreve melhor esse arquivo como ponto de entrada para a aplicação Node.Js, 
  sendo capaz executar o servidor através do comando: \textit{node app.js}
  
  \item[d)] \textit{routes}:
  
  \cite{Pereira:2013} descreve como diretório que mantém todas as rotas da aplicação. 
  Na versão utilizada por \citeonline{Wilson:2013} em seu livro o Express.Js não possui este diretório.
  
  \item[e)] \textit{views}: 
  
  \cite{Wilson:2013} descreve que esta pasta contém os motores de \textit{template} (Jade ou EJS), 
  que são renderizados pelo servidor Express.Js e enviados ao cliente. \cite{Pereira:2013}
  simplifica descrevendo que é um diretório que contém todas as visões renderizadas pelas rotas.
  
\end{compactitem}