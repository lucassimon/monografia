% Nome do Anexo
\chapter{Primeiro Anexo}
\label{primeiro-anexo}
% Para diminuir espa�amento entre o t�tulo e o texto
\vspace{-1.9cm}

% Texto
% Textos ou documentos não elaborados pelo autor.

\begin{compactitem}
  \item[a)] \textit{package.json}:
  
  \citeonline{pereira} diz que este arquivo contém as principais informações sobre a aplicação como: 
  nome, autor, versão, colaboradores, URL, dependências e outros.
  
  \item[b)] \textit{public}:
  
  Pasta pública que armazena código estático como imagens, css e javascript. \cite{pereira}
  
  \item[c)] \textit{app.js}:
  
  \citeonline{wilson} descreve melhor esse arquivo como ponto de entrada para a aplicação Node.Js, 
  sendo capaz executar o servidor através do comando: \textit{node app.js}
  
  \item[d)] \textit{routes}:
  
  \citeonline{pereira} descreve como diretório que mantém todas as rotas da aplicação. 
  Na versão utilizada por \citeonline{wilson} em seu livro o Express.Js não possui este diretório.
  
  \item[e)] \textit{views}: 
  
  \citeonline{wilson} descreve que esta pasta contém os motores de \textit{template} (Jade ou EJS), 
  que são renderizados pelo servidor Express.Js e enviados ao cliente. \citeonline{pereira}
  simplifica descrevendo que é um diretório que contém todas as visões renderizadas pelas rotas.
  
\end{compactitem}

